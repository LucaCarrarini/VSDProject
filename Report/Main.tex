%Preambolo per documento generico
%Per cambiare tipo di documento modificare
%il documentclass.

\documentclass[11pt]{article}
\usepackage[utf8]{inputenc}

%Mettere italian se si vuole scrivere in italiano
\usepackage[english]{babel}


%Aumento dell'interlinea
\usepackage{setspace}
\onehalfspacing

%Pacchetti per simboli matematici
\usepackage{mathtools}
\usepackage{amsfonts}
\usepackage{amsthm}
\newtheorem{theorem}{Theorem}
\usepackage{amsmath}
\usepackage{bm}


%Pacchetti per figure e tabelle
\usepackage{booktabs, caption, graphicx, subfig, float}
\captionsetup{tableposition=top,figureposition=bottom,font=small}

%Per inserire commenti in blocco
%Usage: \begin{comment} ... \end{comment}
\usepackage{comment}

%Definisce colori dei link nel testo e altri parametri
\usepackage{hyperref}
\hypersetup{
pdftitle={VSDProjectReport},%
pdfauthor={Carrarini,Kieffer,Tantucci,Wrona},%
pdfsubject={},%
pdfkeywords={},%
colorlinks=true,%
linkcolor=black,%
linktocpage=true,%
pageanchor=true,
citecolor=black
}

%Definisce in maniera simmetrica i margini di pagina
\usepackage{geometry}
\geometry{a4paper, top=3cm,bottom=3cm,left=3cm,right=3cm,%
			heightrounded}

%Per scrivere in maniera rapida norme e valori assoluti
%Usage: y = \abs{x} ; y = \norma{x}
\DeclarePairedDelimiter{\abs}{\lvert}{\rvert}
\DeclarePairedDelimiter{\norma}{\lVert}{\rVert}

%Testo a caso
%Usage: \lipsum[a-b] | Esempio \lipsum[1-10]
\usepackage{lipsum}

%Cambia lo stile della pagina
%In alto a dx mette numero di pagina
%In alto a sinistra titolo e autore
\usepackage{fancyhdr}
\pagestyle{fancy}
\fancyhf{}
\rhead{\textit{\thepage}}
\lhead{\textit{On KERS implementation}}

%Per inserire codice nativo MATLAB e simili
%Usage: guardare sotto
\usepackage{fancyvrb}

\usepackage[dvipsnames]{xcolor}
\definecolor{sapred}{RGB}{130,36,51} %colore sapienza per i titoli


\begin{document}

\title{\textcolor{sapred}{On KERS implementation}\\\small{Vehicle System Dynamics Project}}
\author{Luca Carrarini\\Federico Kieffer\\Andrea Tantucci\\Andrea Wrona}
%Optional: data. Se non si inserisce il campo data
%LaTeX mette la data del PC automaticamente
%\date{}
\maketitle

\thispagestyle{empty}

%Inizio dei capitoli
\section*{First section}

%Testo a caso
This is a dummy text

\lipsum[1-10]

%Codice per inserire le figure
\begin{comment}
\begin{figure}[H]
\centering
\includegraphics[width=.6\textwidth]{Charts/StepF1}
\caption{Step response of the closed--loop relative to $F_1$}
\label{StepF1}
\end{figure}
\end{comment}


%Codice per inserire il codice
\vspace{4cm}
This is a fancy in--line code
\singlespacing
\begin{Verbatim}[tabsize = 4, frame = lines, numbers = left]
code_here...
	code_here...
code_here...
\end{Verbatim}
\onehalfspacing


\begin{thebibliography}{3}
	
	\bibitem{a}
	First reference.
	
	\bibitem{b}
	Second reference.
	
	\bibitem{c}
	Third reference
	
\end{thebibliography}

\end{document}
